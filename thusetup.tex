% !TeX root = ./thuthesis-example.tex

% 论文基本信息配置

\thusetup{
  %******************************
  % 注意:
  %   1. 配置里面不要出现空行
  %   2. 不需要的配置信息可以删除
  %   3. 建议先阅读文档中所有关于选项的说明
  %******************************
  %
  % 输出格式
  %   选择打印版(print)或用于提交的电子版(electronic),前者会插入空白页以便直接双面打印
  %
  output = print,
  %
  % 标题
  %   可使用“\\”命令手动控制换行
  %
  title  = {ExpertFlow: 实现混合专家模型的低延迟异步推断},
  title* = {ExpertFlow: Enabling Low-Latency Asynchronous Inference for Mixture of Expert Models },
  %
  % 学位
  %   1. 学术型
  %      - 中文
  %        需注明所属的学科门类,例如:
  %        哲学、经济学、法学、教育学、文学、历史学、理学、工学、农学、医学、
  %        军事学、管理学、艺术学
  %      - 英文
  %        博士:Doctor of Philosophy
  %        硕士:
  %          哲学、文学、历史学、法学、教育学、艺术学门类,公共管理学科
  %          填写“Master of Arts“,其它填写“Master of Science”
  %   2. 专业型
  %      直接填写专业学位的名称,例如:
  %      教育博士、工程硕士等
  %      Doctor of Education, Master of Engineering
  %   3. 本科生不需要填写
  %
  degree-name  = {工学硕士},
  degree-name* = {Master of Science},
  %
  % 培养单位
  %   填写所属院系的全名
  %
  department = {计算机科学与技术系},
  %
  % 学科
  %   1. 学术型学位
  %      获得一级学科授权的学科填写一级学科名称,其他填写二级学科名称
  %   2. 工程硕士
  %      工程领域名称
  %   3. 其他专业型学位
  %      不填写此项
  %   4. 本科生填写专业名称,第二学位论文需标注“(第二学位)”
  %
  discipline  = {计算机科学与技术},
  discipline* = {Computer Science and Technology},
  %
  % 姓名
  %
  author  = {欧俊杰},
  author* = {Gabriele Oliaro},
  %
  % 指导教师
  %   中文姓名和职称之间以英文逗号“,”分开,下同
  %
  supervisor  = {翟季冬, 副教授},
  supervisor* = {Professor Zhai Jidong},
  %
  % 副指导教师
  %
  %associate-supervisor  = {TBD, 教授},
  %associate-supervisor* = {Professor TBD},
  %
  % 联合指导教师
  %
  % co-supervisor  = {某某某, 教授},
  % co-supervisor* = {Professor Mou Moumou},
  %
  % 日期
  %   使用 ISO 格式;默认为当前时间
  %
  % date = {2019-07-07},
  %
  % 是否在中文封面后的空白页生成书脊(默认 false)
  %
  include-spine = false,
  %
  % 密级和年限
  %   秘密, 机密, 绝密
  %
  % secret-level = {秘密},
  % secret-year  = {10},
  %
  % 博士后专有部分
  %
  % clc                = {分类号},
  % udc                = {UDC},
  % id                 = {编号},
  % discipline-level-1 = {计算机科学与技术},  % 流动站(一级学科)名称
  % discipline-level-2 = {系统结构},          % 专业(二级学科)名称
  % start-date         = {2011-07-01},        % 研究工作起始时间
}

% 载入所需的宏包

% 定理类环境宏包
\usepackage{amsthm}
% 也可以使用 ntheorem
% \usepackage[amsmath,thmmarks,hyperref]{ntheorem}

\thusetup{
  %
  % 数学字体
  % math-style = GB,  % GB | ISO | TeX
  math-font  = xits,  % stix | xits | libertinus
}

% 可以使用 nomencl 生成符号和缩略语说明
% \usepackage{nomencl}
% \makenomenclature

% 表格加脚注
\usepackage{threeparttable}

% 表格中支持跨行
\usepackage{multirow}

% 固定宽度的表格。
% \usepackage{tabularx}

% 跨页表格
\usepackage{longtable}

% 算法
\usepackage{algorithm}
\usepackage{algorithmicx}
\usepackage{algpseudocode}
\usepackage{cancel}
%\usepackage{markdown}
\usepackage{tikz}
\newcommand*\circled[1]{\tikz[baseline=(char.base)]{
    \node[shape=circle,draw,inner sep=2pt] (char) {#1};}}

\usepackage{listings}
\definecolor{mGreen}{rgb}{0,0.6,0}
\definecolor{mGray}{rgb}{0.5,0.5,0.5}
\definecolor{mPurple}{rgb}{0.58,0,0.82}
\definecolor{backgroundColour}{rgb}{0.95,0.95,0.92}
\definecolor{darkred}{rgb}{0.5, 0.0, 0.13}
\definecolor{carnelian}{rgb}{0.7, 0.11, 0.11}

\lstdefinestyle{CStyle}{
    % backgroundcolor=\color{backgroundColour},   
    commentstyle=\color{mGreen},
    keywordstyle=\color{magenta},
    emphstyle = \color{blue!60!black}\bfseries,
    numberstyle=\tiny\color{mGray},
    stringstyle=\color{mPurple},
    basicstyle=\ttfamily\small,
    columns=flexible,
    breakatwhitespace=false,         
    breaklines=true,                 
    captionpos=b,                    
    keepspaces=true,                 
    numbers=left,                    
    numbersep=5pt,                  
    showspaces=false,                
    showstringspaces=false,
    showtabs=false,                  
    tabsize=2,
    language=C,
}

% algorithm-related custom code
\definecolor{commentcolor}{gray}{0.5}
\renewcommand{\algorithmicrequire}{\textbf{Require:}\unskip}
\renewcommand{\algorithmicensure}{\textbf{Input:}\unskip}
\renewcommand{\algorithmiccomment}[1]{\hfill$\vartriangleright${\color{commentcolor}{\textit{#1}}}}


\algnewcommand\algorithmicto{\textbf{to}}
\algrenewtext{For}[3]%
{\algorithmicfor\ #1 \gets #2 \algorithmicto\ #3 \algorithmicdo}

\algblock{ParallelFor}{EndParallelFor}
% customising the new block
\algnewcommand\algorithmicparfor{\textbf{parallelfor}}
\algnewcommand\algorithmicpardo{\textbf{do}}
\algnewcommand\algorithmicendparfor{\textbf{end\ parallelfor}}
\algrenewtext{ParallelFor}[1]{\algorithmicparfor\ #1\ \algorithmicpardo}
\algrenewtext{EndParallelFor}{\algorithmicendparfor}
%\algnewcommand\algorithmicto{\textbf{to}}
\algrenewtext{ParallelFor}[3]%
{\algorithmicparfor\ #1 \gets #2 \algorithmicto\ #3 \algorithmicdo}


% 量和单位
\usepackage{siunitx}

% 参考文献使用 BibTeX + natbib 宏包
% 顺序编码制
\usepackage[sort]{natbib}
\bibliographystyle{thuthesis-numeric}

% 著者-出版年制
% \usepackage{natbib}
% \bibliographystyle{thuthesis-author-year}

% 本科生参考文献的著录格式
% \usepackage[sort]{natbib}
% \bibliographystyle{thuthesis-bachelor}

% 参考文献使用 BibLaTeX 宏包
% \usepackage[style=thuthesis-numeric]{biblatex}
% \usepackage[style=thuthesis-author-year]{biblatex}
% \usepackage[style=apa]{biblatex}
% \usepackage[style=mla-new]{biblatex}
% 声明 BibLaTeX 的数据库
% \addbibresource{ref/refs.bib}

% 定义所有的图片文件在 figures 子目录下
\graphicspath{{figures/}}

% 数学命令
\makeatletter
\newcommand\dif{%  % 微分符号
  \mathop{}\!%
  \ifthu@math@style@TeX
    d%
  \else
    \mathrm{d}%
  \fi
}
\makeatother

\newcommand{\Project}{ExpertFlow }

% hyperref 宏包在最后调用
\usepackage{hyperref}
